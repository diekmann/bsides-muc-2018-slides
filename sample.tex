%!TEX program = xelatex
\documentclass{beamer}

\usepackage{blindtext}

\usetheme{Execushares}

\title{A New Beamer Theme That Works Well and Looks Great: Execushares}
\subtitle{A custom modern minimalist Beamer theme designed from scratch}
\author{Kenton Hamaluik}
\date{June 1, 2014}

\setcounter{showSlideNumbers}{1}




\newcommand{\hairspace}{\hspace{1pt}}
\newcommand{\eg}{\mbox{e.\hairspace{}g.,} }  % z.B.
\newcommand{\ie}{\mbox{i.\hairspace{}e.,} }  % d.h.
\newcommand{\Ie}{\mbox{I.\hairspace{}e.,} }  % d.h.
\newcommand{\cf}{\mbox{cf.}\ }
\newcommand{\etal}{\mbox{et~al.}\ }

\usepackage{fancyvrb}

\usepackage{IEEEtrantools}

%https://tex.stackexchange.com/questions/220820/itemize-list-inside-a-tikzpicture-node
\usepackage{varwidth}

% math typesetting
% free variables
\newcommand{\mvar}[1]{\ensuremath{\mathit{#1}}}
% definitions and constants
\newcommand{\mdef}[1]{\ensuremath{\mathsf{#1}}}
% executable functions
\newcommand{\mfun}[1]{\mdef{#1}}
%datatype constructor (may appear in text)
\newcommand\mconstr[1]{\mdef{#1}}
%math control (if then else)
\newcommand\mctrl[1]{\ensuremath{\mathbf{#1}}}


\newcommand{\BigO}{\mathcal{O}}


\DeclareMathSymbol{\mlq}{\mathord}{operators}{``}
\DeclareMathSymbol{\mrq}{\mathord}{operators}{`'}


\newcommand\mdoubleplus{\ensuremath{\mathbin{+\mkern-10mu+}}}
\newcommand\lstapp{\ensuremath{\mathbin{:\mkern-1mu:\mkern-1mu:}}} %list append
\newcommand\lstcons{\ensuremath{\mathbin{:\mkern-1mu:}}} %list append


\newcommand{\free}[1]{\textcolor{DarkerBlue}{#1}}
\newcommand{\bound}[1]{\textcolor{DarkGreen}{#1}}
%\newcommand{\boundi}[1]{\textcolor{Green}{#1}}


\newcommand*\circled[1]{\tikz[baseline=-3pt, scale=0.5, every node/.style={scale=0.5}]{\node[shape=circle,draw,inner sep=1pt,minimum size=16pt] (char) {#1};}}

\newcommand\allow{\ensuremath{\textnormal{\circled{\large \textbf{\checkmark}}}}}
\newcommand\deny{\ensuremath{\textnormal{\circled{\large \textbf{\texttimes}}}}}
\newcommand\undecided{\ensuremath{\textnormal{\circled{\textnormal{\large \textbf{?}}}}}}

\newcommand\matchop[1]{\ensuremath{\mfun{match}\ {#1}}}
\newcommand\matches[1]{\ensuremath{\matchop\gamma \ #1 \ p}}
\newcommand\nmatches[1]{\ensuremath{\neg\; \matchop\gamma \ #1 \ p}}
\newcommand\bigstep[3]{\ensuremath{\free{\Gamma},\free{\gamma},\free{p} \vdash\big\langle #1,\; #2 \big\rangle \Rightarrow #3}}






\begin{document}
	\setcounter{showProgressBar}{0}
	\setcounter{showSlideNumbers}{0}

	\frame{\titlepage}

	\begin{frame}
		\frametitle{Contents}
		\begin{enumerate}
			\item Introduction \\ \textcolor{ExecusharesGrey}{\footnotesize\hspace{1em} The reasoning and background behind this theme}
			\item Lorem Text  \\ \textcolor{ExecusharesGrey}{\footnotesize\hspace{1em} Just some Lorem Ipsum for filler}
			\item Conclusions \\ \textcolor{ExecusharesGrey}{\footnotesize\hspace{1em} Some closing thoughts}
		\end{enumerate}
	\end{frame}

	\setcounter{framenumber}{0}
	\setcounter{showProgressBar}{1}
	\setcounter{showSlideNumbers}{1}
	\section{Introduction}
		\begin{frame}
			\frametitle{Why Beamer?}
			\begin{enumerate}
				\item LaTeX is great!
				\item Beamer is easy to use!
				\item Why not?
			\end{enumerate}
		\end{frame}

		\begin{frame}
			\frametitle{Why Custom Themes?}
			\begin{itemize}
				\item The default Beamer themes are outdated and visually displeasing
				\item There aren't many Beamer themes readily available online
				\item Making custom Beamer themes is easy!
			\end{itemize}
		\end{frame}
		

		\begin{frame}
			\frametitle{Theorem!}
				Assumes	
	\begin{itemize}
		\item Unfolded $\free{\mvar{rs}}$ for $\free{\Gamma}$
		\item $\free{\mvar{p}}$ is \texttt{NEW}% and has \texttt{SYN} set (if TCP)
		\item $\bigstep{\free{\mvar{rs}}}{\undecided}{\allow}$
		\item Let $(\free{V}, \free{E}) = \mfun{matrix}\ (\mdef{iifce}\ \free{p}, \mdef{oifce}\ \free{p}, \mdef{prot}\ \free{p}, \mdef{sport}\ \free{p}, \mdef{dport}\ \free{p})\ (\mfun{simplify}\ \free{\mvar{rs}})$
	\end{itemize}
	Shows%
	\begin{center}%
		\vskip-3ex%
	    \begin{IEEEeqnarray*}{l}
		\exists \bound{\mvar{s}_\text{repr}}\ \bound{\mvar{d}_\text{repr}}\ \bound{\mvar{s}_\text{range}}\ \bound{\mvar{d}_\text{range}}.\ \ (\bound{\mvar{s}_\text{repr}}, \bound{\mvar{d}_\text{repr}}) \in  \mdef{set}\ \free{\mvar{E}} \  \wedge  \\
		\qquad (\mfun{map\_of}\ \free{\mvar{V}})\ \bound{\mvar{s}_\text{repr}} = \mdef{Some}\ \bound{\mvar{s}_\text{range}} \ \wedge \ (\mdef{src}\ \free{\mvar{p}}) \in \bound{\mvar{s}_\text{range}} \ \wedge \\
		\qquad (\mfun{map\_of}\ \free{\mvar{V}})\ \bound{\mvar{d}_\text{repr}} = \mdef{Some}\ \bound{\mvar{d}_\text{range}} \ \wedge \ (\mdef{dst}\ \free{\mvar{p}}) \in \bound{\mvar{d}_\text{range}}
		\end{IEEEeqnarray*}
	\end{center}
	\medskip
	%Reads: If the firewall accepts a packet, we can look up source and destination IP in the graph.
	
		\end{frame}

	\section{Lorem Ipsum}
		\begin{frame}
			\frametitle{Lorem 1}
			\blindtext
		\end{frame}

		\begin{frame}
			\frametitle{Lorem 2}
			\blindtext
		\end{frame}

		\begin{frame}
			\frametitle{Lorem 3}
			\blindtext
		\end{frame}

	\section{Conclusions}
		\begin{frame}
			\frametitle{Closing Thoughts}
			\begin{itemize}
				\item Woo, Beamer!
			\end{itemize}
		\end{frame}
	
	\appendix
	\backupbegin
	  \begin{frame}
	    \frametitle{Backup slide 1}
	    \blindtext
	  \end{frame}
	\backupend

\end{document}
